\documentclass[10pt,a4paper,twoside,spanish]{article}	%Indica la clase de documento

\pagestyle{plain}

\usepackage[spanish,english]{babel}
\selectlanguage{spanish}
\usepackage[utf8]{inputenc}


\usepackage{amsmath,amssymb,amsfonts,latexsym,cancel}  
\usepackage[table]{xcolor} 				%Para poner color en las tablas
\usepackage{float}
\setlength\unitlength{1mm}
%*************** Agregados de formato********************
%\usepackage{shortlst}					% Intems cortos
\usepackage{multicol}					% Permite dividir en columnas
\usepackage{wrapfig}					% Texto al lado de las figuras
\usepackage[rftl]{floatflt}
\usepackage{rotating}					% Rotar elementos
\usepackage{subfig}					% Subfiguras
\usepackage{anysize}					% Configurar margenes
\marginsize{2 cm}{2 cm}{1 cm}{2 cm}
\usepackage{multirow}
%************************************************************
%***** Encabezado y pie de pagina ***********************
\usepackage{fancyhdr}
\pagestyle{fancy}
\fancyhf{}

\fancyhead[HR]{Universidad de Buenos Aires} % Números de página en las esquinas de los encabezados
\fancyhead[HL]{Carrera de Especialización en Sistemas Embebidos}

\fancyfoot[FL]{Arquitectura de Microprocesadores Numérico} 
\fancyfoot[FR]{\thepage} 

\renewcommand{\headrulewidth}{0.6pt} % Ancho de la línea horizontal bajo el encabezado
\renewcommand{\footrulewidth}{0.6pt} % Ancho de la línea horizontal sobre el pie (que en este ejemplo está vacío)
\setlength{\headheight}{1.5\headheight} % Aumenta la altura del encabezado en una vez y media
%************************************************************
% Define el formato del título
\newcommand{\titulo }[1]
{\begin{center}
	\huge	\textbf{#1}\\
\end{center}
}
%************************************************************
%************************************************************

%%%%%%%%%%%%%%%%%%%%%%%%%%%%%%%%%%%%%%%%%%%%%%%%%%%%%%%%%%%%%%%%%%%%%%%%
\begin{document}

\section*{Preguntas orientadoras}

\textbf{Describa brevemente los diferentes perfiles de familias de
microprocesadores/microcontroladores de ARM. Explique alguna de sus diferencias
características.}

\subsubsection*{Cortex A:}
Son procesadores de alto rendimiento optimizados para aplicaciones que emplean un sistema operativo de propósito general en sistemas embebidos de alta performance. 

Su denominación ``A" proviene de \textit{Application}, se pueden encontrar en dispositivos como celulares o tables.  
 
Se destaca la optimización para ejecutar diversas aplicaciones al mismo tiempo a costa de una disminución del tiempo de respuesta de las mismas, aspecto secundario en dispositivo de usuario. 


\subsubsection*{Cortex R:}

Son procesadores orientados a sistemas de tiempo real donde prima la
necesidad de implementar soluciones con requerimientos temporales estrictos.

En los sistemas de tiempo real la respuesta a eventos o estímulos debe ser en un tiempo acotado y preestablecido comportándose de forma determinística.  

Su denominación ``R" proviene de \textit{Real Time}, se pueden encontrar en sistemas críticos, como dispositivos médicos o sistemas de operación automovilísticos.


\subsubsection*{Cortex M:}


Son procesadores orientados a dispositivos de consumo masivo y sistemas
embebidos compactos. Son procesadores de uso general diseñados para alta densidad de código y con gran cantidad de periféricos. Su denominación ``M" proviene de \textit{Microcontrollers}, se pueden encontrar en dispositivos como celulares o tables. 

%\section*{Cortex M}
%
%\subsection*{1. Describa brevemente las diferencias entre las familias de procesadores Cortex M0, M3 y M4}
%
%\subsection*{2. ¿Por qué se dice que el set de instrucciones Thumb permite mayor densidad de código? Explique}
%
%\subsection*{3. ¿Qué entiende por arquitectura load-store? ¿Qué tipo de instrucciones no posee este tipo de arquitectura?}

\end{document}